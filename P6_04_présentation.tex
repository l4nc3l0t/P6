\documentclass[8pt,aspectratio=169,hyperref={unicode=true}]{beamer}

\usefonttheme{serif}
\usepackage{fontspec}
	\setmainfont{TeX Gyre Heros}
\usepackage{unicode-math}
\usepackage{lualatex-math}
	\setmathfont{TeX Gyre Termes Math}
\usepackage{polyglossia}
\setdefaultlanguage[frenchpart=false]{french}
\setotherlanguage{english}
%\usepackage{microtype}
\usepackage[locale = FR,
            separate-uncertainty,
            multi-part-units = single,
            range-units = single]{siunitx}
	\DeclareSIUnit\an{an}
  \DeclareSIUnit{\octet}{o}
\usepackage{amsmath}
\usepackage{amsfonts}
\usepackage{amssymb}
\usepackage{array}
\usepackage{graphicx}
\graphicspath{{./Figures/}}
\usepackage{booktabs}
\usepackage{tabularx}
\usepackage{multirow}
\usepackage{multicol}
    \newcolumntype{L}{>{\raggedright\arraybackslash}X}
    \newcolumntype{R}{>{\raggedleft\arraybackslash}X}
\usepackage{tikz}
\usetikzlibrary{graphs, graphdrawing, arrows.meta} \usegdlibrary{layered, trees}
\usetikzlibrary{overlay-beamer-styles}
\usepackage{subcaption}
\usepackage[]{animate}
\usepackage{float}
\usepackage{csquotes}

\usetheme[secheader]{Boadilla}
\usecolortheme{seagull}
\setbeamertemplate{enumerate items}[default]
\setbeamertemplate{itemize items}{-}
\setbeamertemplate{navigation symbols}{}
\setbeamertemplate{bibliography item}{}
\setbeamerfont{framesubtitle}{size=\large}
\setbeamertemplate{section in toc}[sections numbered]
%\setbeamertemplate{subsection in toc}[subsections numbered]

\title[Classifiez des biens de consommation]{Projet 6 : Classifiez des biens de consommation}
\author[Lancelot \textsc{Leclercq}]{Lancelot \textsc{Leclercq}} 
\institute[]{}
\date[]{\small{7 mars 2022}}

\AtBeginSection[]{
  \begin{frame}
  \vfill
  \centering
    \usebeamerfont{title}\insertsectionhead\par%
  \vfill
  \end{frame}
}

\begin{document}
\setbeamercolor{background canvas}{bg=gray!20}
\begin{frame}[plain]
    \titlepage
\end{frame}

\begin{frame}{Sommaire}
    \Large
    \begin{columns}
        \begin{column}{.7\textwidth}
            \tableofcontents[hideallsubsections]
        \end{column}
    \end{columns}
\end{frame}

\section{Introduction}
\subsection{Problématique}
\begin{frame}{\insertsubsection}
    \begin{itemize}
        \item L'entreprise Place de marché est un marketplace e-commerce
              \begin{itemize}
                  \item Vendeurs proposent des articles à des acheteurs en postant une photo et une description
                  \item[]
                  \item Attribution de la catégorie d'un article effectuée manuellement par les vendeurs $\Longrightarrow$ peu fiable
              \end{itemize}
        \item[]
        \item Objectif
              \begin{itemize}
                  \item Améliorer l'expérience utilisateur des vendeurs et des acheteurs
                  \item[]
                  \item Automatisation de l'attribution d'une catégorie
                  \item[]
                  \item Étude de la faisabilité d'un moteur de classification
              \end{itemize}
    \end{itemize}
\end{frame}

\subsection{Jeu de données}
\begin{frame}{\insertsubsection}
    \begin{columns}
        \begin{column}{.5\textwidth}
            \begin{itemize}
                \item Jeu de données textuelles
                      \begin{itemize}
                          \item Nom, prix, description, note, pour chaque objet
                      \end{itemize}
            \end{itemize}
            \begin{table}
                \input{./Tableaux/dataCol.tex}
            \end{table}
        \end{column}
        \begin{column}{.5\textwidth}
            \begin{itemize}
                \item Jeu d'images
                      \begin{itemize}
                          \item Nous avons une image par objet
                      \end{itemize}
            \end{itemize}
            \begin{figure}
                \includegraphics[width=\textwidth]{testIMG.pdf}
                \caption{Exemple d'image associée à un objet (ici des rideaux)}
            \end{figure}
        \end{column}
    \end{columns}
\end{frame}

\section{Classification des descriptions textuelles}
\subsection{Méthode}
\subsubsection{Nettoyage et création de "bag of words"}
\begin{frame}{\insertsubsection}{\insertsubsubsection}
    \begin{itemize}
        \item Nettoyage:
              \begin{itemize}
                  \item Retrait des chiffres et caractères spéciaux,
                  \item Retrait de la ponctuation,
                  \item Uniformisation de la casse
              \end{itemize}
        \item Tokenisation
              \begin{itemize}
                  \item conservation des mots pertinents à partir de listes de "stopwords", des mots très récurrents à supprimer
              \end{itemize}
        \item Lemmatisation
        \begin{itemize}
            \item Similaire à la tokenisation avec la suppression des terminaisons des mots
            \item Permet d'uniformiser les variations singulier/pluriel, masculin/féminin
        \end{itemize}
        \item Racinisation (Stemmatisation)
        \begin{itemize}
            \item Similaire à la lemmatisation avec conservation de la racine des mots
            \item Permet d'uniformiser les variations de vocabulaires en regroupant les mots ayant les mêmes racines
        \end{itemize}
    \end{itemize}
    \begin{table}
        \scriptsize
        \input{./Tableaux/CompareTxt.tex}
    \end{table}
\end{frame}
\subsubsection{Vectorisation}

\subsection{Comparaison}

\section{Classification des images}
\subsection{Méthode}

\subsection{Comparaison}

\section{Classification de l'ensemble des données}

\section{Conclusion}
\end{document}