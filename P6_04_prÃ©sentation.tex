\documentclass[8pt,aspectratio=169,hyperref={unicode=true}]{beamer}

\usefonttheme{serif}
\usepackage{fontspec}
	\setmainfont{TeX Gyre Heros}
\usepackage{unicode-math}
\usepackage{lualatex-math}
	\setmathfont{TeX Gyre Termes Math}
\usepackage{polyglossia}
\setdefaultlanguage[frenchpart=false]{french}
\setotherlanguage{english}
%\usepackage{microtype}
\usepackage[locale = FR,
            separate-uncertainty,
            multi-part-units = single,
            range-units = single]{siunitx}
	\DeclareSIUnit\an{an}
  \DeclareSIUnit{\octet}{o}
\usepackage{amsmath}
\usepackage{amsfonts}
\usepackage{amssymb}
\usepackage{array}
\usepackage{graphicx}
\graphicspath{{./Figures/}}
\usepackage{booktabs}
\usepackage{tabularx}
\usepackage{multirow}
\usepackage{multicol}
    \newcolumntype{L}{>{\raggedright\arraybackslash}X}
    \newcolumntype{R}{>{\raggedleft\arraybackslash}X}
\usepackage{xcolor}
\usepackage{tikz}
\usetikzlibrary{graphs, graphdrawing, arrows.meta} \usegdlibrary{layered, trees}
\usetikzlibrary{overlay-beamer-styles}
\usepackage{subcaption}
\usepackage[]{animate}
\usepackage{float}
\usepackage{csquotes}

\usetheme[secheader]{Boadilla}
\usecolortheme{seagull}
\setbeamertemplate{enumerate items}[default]
\setbeamertemplate{itemize items}{-}
\setbeamertemplate{navigation symbols}{}
\setbeamertemplate{bibliography item}{}
\setbeamerfont{framesubtitle}{size=\large}
\setbeamertemplate{section in toc}[sections numbered]
%\setbeamertemplate{subsection in toc}[subsections numbered]

\title[Classifiez des biens de consommation]{Projet 6 : Classifiez des biens de consommation}
\author[Lancelot \textsc{Leclercq}]{Lancelot \textsc{Leclercq}} 
\institute[]{}
\date[]{\small{7 mars 2022}}

\AtBeginSection[]{
  \begin{frame}
  \vfill
  \centering
    \usebeamerfont{title}\insertsectionhead\par%
  \vfill
  \end{frame}
}

\begin{document}
\setbeamercolor{background canvas}{bg=gray!20}
\begin{frame}[plain]
    \titlepage
\end{frame}

\begin{frame}{Sommaire}
    \Large
    \begin{columns}
        \begin{column}{.7\textwidth}
            \tableofcontents[hideallsubsections]
        \end{column}
    \end{columns}
\end{frame}

\section{Introduction}
\subsection{Problématique}
\begin{frame}{\insertsubsection}
    \begin{itemize}
        \item L'entreprise Place de marché est un marketplace e-commerce
              \begin{itemize}
                  \item Vendeurs proposent des articles à des acheteurs en postant une photo et une description
                  \item[]
                  \item Attribution de la catégorie d'un article effectuée manuellement par les vendeurs $\Longrightarrow$ peu fiable
              \end{itemize}
        \item[]
        \item Objectif
              \begin{itemize}
                  \item Améliorer l'expérience utilisateur des vendeurs et des acheteurs
                  \item[]
                  \item Automatisation de l'attribution d'une catégorie
                  \item[]
                  \item Étude de la faisabilité d'un moteur de classification
              \end{itemize}
    \end{itemize}
\end{frame}

\subsection{Jeu de données}
\begin{frame}{\insertsubsection}
    \begin{columns}
        \begin{column}{.5\textwidth}
            \begin{itemize}
                \item Jeu de données textuelles
                      \begin{itemize}
                          \item Nom, prix, description, note, pour chaque objet
                      \end{itemize}
            \end{itemize}
            \begin{table}
                \only<1>{\input{./Tableaux/dataCol.tex}}
                \only<2>{\input{./Tableaux/dataColRed.tex}}
            \end{table}
        \end{column}
        \begin{column}{.5\textwidth}
            \begin{itemize}
                \item Jeu d'images
                      \begin{itemize}
                          \item Nous avons une image par objet
                      \end{itemize}
            \end{itemize}
            \begin{figure}
                \includegraphics[width=\textwidth]{testIMG.pdf}
                \caption{Exemple d'image associée à un objet (ici des rideaux)}
            \end{figure}
        \end{column}
    \end{columns}
\end{frame}

\section{Classification des descriptions textuelles}
\subsection{Méthode : Nettoyage et création de "bag of words"}
\subsubsection{Sélection des stopwords personalisés}
\begin{frame}{\insertsubsection}{\insertsubsubsection}
    \begin{columns}
        \begin{column}{.4\textwidth}
            \begin{itemize}
                \item Nettoyage:
                      \begin{itemize}
                          \item Retrait des chiffres et caractères spéciaux,
                          \item Retrait de la ponctuation,
                          \item Uniformisation de la casse
                      \end{itemize}
                \item Tokenisation
                      \begin{itemize}
                          \item conservation des mots pertinents à partir de listes de "stopwords", des mots très récurrents à supprimer
                      \end{itemize}
                \item Lemmatisation
                      \begin{itemize}
                          \item Similaire à la tokenisation avec la suppression des terminaisons des mots
                          \item Permet d'uniformiser les variations singulier/pluriel, masculin/féminin
                      \end{itemize}
                \item Racinisation (Stemmatisation)
                      \begin{itemize}
                          \item Similaire à la lemmatisation avec conservation de la racine des mots
                          \item Permet d'uniformiser les variations de vocabulaires en regroupant les mots ayant les mêmes racines
                      \end{itemize}
            \end{itemize}
        \end{column}
        \begin{column}{.6\textwidth}
            \includegraphics[width=\textwidth]{FreqTok50.pdf}
            \includegraphics[width=\textwidth]{FreqLem50.pdf}
            \includegraphics[width=\textwidth]{FreqStem50.pdf}
        \end{column}
    \end{columns}
\end{frame}

\subsubsection{Comparaison des différents traitements du texte}
\begin{frame}{\insertsubsection}{\insertsubsubsection}
    \begin{table}
        \scriptsize
        \input{./Tableaux/CompareTxt.tex}
    \end{table}
\end{frame}

\subsection{Comparaison : validation croisée}
\begin{frame}{\insertsubsection}
    \begin{columns}
        \begin{column}{.45\textwidth}
            \begin{itemize}
                \item Essais pour les 3 types de nettoyages
                      \begin{itemize}
                          \item tokenisation
                          \item lemmatisation
                          \item racinisation
                      \end{itemize}
                \item[]
                \item Essais de trois vectoriseurs :
                      \begin{itemize}
                          \item TfidfVectorizer
                          \item CountVectorizer
                          \item HashingVectorizer
                      \end{itemize}
                \item[]
                \item Essais de différents découpages :
                      \begin{itemize}
                          \item mono-grams
                          \item mono et bigrams
                          \item mono, bi et trigrams
                      \end{itemize}
            \end{itemize}
        \end{column}
        \begin{column}{.55\textwidth}
            \includegraphics[width=\textwidth]{CompareScores.pdf}
        \end{column}
    \end{columns}
\end{frame}

\subsection{Meilleure classification}
\begin{frame}{\insertsubsection}
    \begin{columns}
        \begin{column}{.4\textwidth}
            \begin{itemize}
                \item Meilleur score en utilisant:
                      \begin{itemize}
                          \item les racines des mots
                          \item des monograms
                          \item le vectoriseur tf-idf
                      \end{itemize}
                \item ARI = 0,525
            \end{itemize}
        \end{column}
        \begin{column}{.6\textwidth}
            \includegraphics[width=\textwidth]{HeatmapLabels40RacinesTfidfVectorizerMono.pdf}
        \end{column}
    \end{columns}
\end{frame}

\subsection{Visualisation de la meilleure classification}
\begin{frame}{\insertsubsection}
    \begin{columns}
        \begin{column}{.5\textwidth}
            \includegraphics[width=\textwidth]{tsne40RacinesTfidfVectorizerMono.pdf}
        \end{column}
        \begin{column}{.5\textwidth}
            \includegraphics[width=\textwidth]{kmean40RacinesTfidfVectorizerMono.pdf}
        \end{column}
    \end{columns}
\end{frame}

\section{Classification des images}
\subsection{Méthode : Nettoyage}
\subsubsection{Traitement des images pour les algorithmes SIFT et ORB}
\begin{frame}{\insertsubsection}{\insertsubsubsection}
    \begin{columns}
        \begin{column}{.5\textwidth}
            \includegraphics[width=\textwidth]{imgBW.pdf}
        \end{column}
        \begin{column}{.5\textwidth}
            \begin{itemize}
                \item[$\longleftarrow$] Utilisation de niveaux de gris
                \item[$\swarrow$] Équalisation de l'histogramme
                \item[$\downarrow$] Filtration de l'image
            \end{itemize}
        \end{column}
    \end{columns}
    \begin{columns}
        \begin{column}{.5\textwidth}
            \includegraphics[width=\textwidth]{imgBWCLAHE.pdf}
        \end{column}
        \begin{column}{.5\textwidth}
            \includegraphics[width=\textwidth]{imgBWCLAHENlMDFilt.pdf}
        \end{column}
    \end{columns}
\end{frame}

\subsection{Comparaison}
\begin{frame}{\insertsubsection}
    \begin{columns}
        \begin{column}{.4\textwidth}
            \begin{itemize}
                \item Essais avec 5 algorithmes:
                      \begin{itemize}
                          \item SIFT
                          \item ORB
                          \item VGG16
                          \item Xeption
                          \item InceptionV3
                      \end{itemize}
            \end{itemize}
        \end{column}
        \begin{column}{.6\textwidth}
            \includegraphics[width=\textwidth]{CompareIMGScores.pdf}
        \end{column}
    \end{columns}
\end{frame}

\subsection{Meilleure classification}
\begin{frame}{\insertsubsection}
    \begin{columns}
        \begin{column}{.4\textwidth}
            \begin{itemize}
                \item Meilleure classification avec : InceptionV3
                \item ARI = 0,535
            \end{itemize}
        \end{column}
        \begin{column}{.6\textwidth}
            \includegraphics[width=\textwidth]{HeatmapLabels40IV3.pdf}
        \end{column}
    \end{columns}
\end{frame}

\subsection{Visualisation de la meilleure classification}
\begin{frame}{\insertsubsection}
    \begin{columns}
        \begin{column}{.5\textwidth}
            \includegraphics[width=\textwidth]{tsne50IV3.pdf}
        \end{column}
        \begin{column}{.5\textwidth}
            \includegraphics[width=\textwidth]{kmean50IV3.pdf}
        \end{column}
    \end{columns}
\end{frame}


\section{Classification de l'ensemble des données}

\subsection{Méthode}
\begin{frame}{\insertsubsection}
    \begin{itemize}
        \item Utilisation de la matrice "bag of words" pour la meilleure classification du texte (racines) avec la matrice "bag of visual words" pour la meilleure classification des images (InceptionV3)
        \item[]
        \item Essais avec plusieurs valeurs de composantes utilisées pour la PCA afin d'affiner la classification
    \end{itemize}
\end{frame}

\subsection{Comparaison}
\begin{frame}{\insertsubsection}
    \begin{columns}
        \begin{column}{.4\textwidth}
            \begin{itemize}
                \item Essais pour différentes valeurs de composantes utilisée pour la PCA
            \end{itemize}
        \end{column}
        \begin{column}{.6\textwidth}
            \includegraphics[width=\textwidth]{CompareFullScores.pdf}
        \end{column}
    \end{columns}
\end{frame}

\subsection{Meilleure classification}
\begin{frame}{\insertsubsection}
    \begin{columns}
        \begin{column}{.4\textwidth}
            \begin{itemize}
                \item Meilleure classification avec 150 composantes
                \item ARI = 0,536
            \end{itemize}
        \end{column}
        \begin{column}{.6\textwidth}
            \includegraphics[width=\textwidth]{HeatmapLabels60StemtfidfMonoIV3150.pdf}
        \end{column}
    \end{columns}
\end{frame}

\subsection{Visualisation de la meilleure classification}
\begin{frame}{\insertsubsection}
    \begin{columns}
        \begin{column}{.5\textwidth}
            \includegraphics[width=\textwidth]{tsne60StemtfidfMonoIV3150.pdf}
        \end{column}
        \begin{column}{.5\textwidth}
            \includegraphics[width=\textwidth]{kmean60StemtfidfMonoIV3150.pdf}
        \end{column}
    \end{columns}
\end{frame}

\section{Comparaison des différentes classifications}
\subsection{Matrices de corrélations}
\begin{frame}{\insertsection}{\insertsubsection}
    \begin{columns}
        \begin{column}{.33\textwidth}
            Classification à partir des données textuelles uniquement
            \includegraphics[width=\textwidth]{HeatmapLabels40RacinesTfidfVectorizerMono.pdf}
        \end{column}
        \begin{column}{.33\textwidth}
            Classification à partir des images uniquements
            \includegraphics[width=\textwidth]{HeatmapLabels40IV3.pdf}
        \end{column}
        \begin{column}{.33\textwidth}
            Classification à partir de toutes les données
            \includegraphics[width=\textwidth]{HeatmapLabels60StemtfidfMonoIV3150.pdf}
        \end{column}
    \end{columns}
    \begin{itemize}
        \item[]
        \item Les images permettent d'éviter les classifications aberrantes comme la confusion entre produits informatiques et produits pour bébé avec le texte uniquement
        \item Confusion entre les fournitures de maison et les produits pour bébé avec les données d'image uniquement qui se répercute lors de l'utilisation des données complètes
        \item Difficulté à classifier les produits de décoration et de fête (Home Decor \& Festive Needs)
    \end{itemize}
\end{frame}


\section{Conclusion}
\begin{frame}
    \begin{itemize}
        \item Classification des données séparées moyenne
              \begin{itemize}
                  \item ARI légèrement supérieure à 0,5 dans les deux cas
              \end{itemize}
        \item[]
        \item La combinaison des deux types de données n'améliore pas significativement le score mais evite les classements aberrants (mélanges entre informatique et produits pour bébé avec le texte uniquement)
        \item[]
        \item Il pourrai être intéressant d'effectuer une classification supervisée à partir des catégories déjà assignées afin d'afiner le classement
        \item[]
        \item Il pourrait aussi utiliser des mots clés dans les descriptions qui permettraient de positionner sans ambiguité certains produits difficiles à classer avec les images
        \item[]
        \item Il peut aussi revoir les catégories afin d'améliorer le classement automatique
    \end{itemize}
\end{frame}
\end{document}